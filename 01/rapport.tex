\title{IDATT2101 Øving 1}

\author {
    Jakob Grønhaug \\
    jakobkg@stud.ntnu.no}

\hypertarget{algoritme}{%
\section{Algoritme}\label{algoritme}}

Algoritmen som er implementert for denne oppgaven er basert på å
analysere differanse mellom relative aksje-priser, og kan beskrives som
følger:

\begin{verbatim}
Gitt en liste med kursendringer per dag
For hver dag (i fra 1 til n)
    Iterer gjennom alle påfølgende dager (j fra i+1 til n)
        Beregn pris-differansen mellom dag i og dag j
        Hvis dette er en ny største differanse,
            noter dagene for kjøp (i) og salg (j), og differansen
\end{verbatim}

I denne algoritmen vil den ytterste løkken utføres n ganger, og den
innerste løkken vil utføres mellom n-1 og 1 ganger basert på hvor mange
dager det er igjen i listen over kursendringer (n - i). Dette gir en
teoretisk kompleksitet på O(n\^{}2).

\hypertarget{implementasjon}{%
\section{Implementasjon}\label{implementasjon}}

En implementasjon av algoritmen i Rust medfølger i filen ``aksjer.rs'',
og et skript for testing av dette i filen ``benchmark''. Skriptet
``benchmark'' er basert på bash, og er kun testet i Linux-miljø.
Kompilering av implementasjonen krever en installert Rust-kompilator
(rustc). En forhånds-kompilert kjørbar programfil medfølger for
enkelhets skyld, kalt ``aksjer''. Denne er kompilert for Linux-miljø, og
ikke testet i andre miljøer.

Implementasjonen i ``aksjer.rs'' inneholder også generering av
tilfeldige kursendringer i intervallet {[}-10, 10{]} som brukes som
test-data for algoritmen, tidtaking av algoritmen og enkel presentasjon
av resultatet av algoritmen og tidtakingen til bruker. Den overordnede
strukturen til implementasjonen er

\begin{verbatim}
Generer n tilfeldige kursendringer
Klargjør tidtaking
Utfør algoritmen på de genererte kursendringene
Avslutt tidtaking
Presenter resultat
\end{verbatim}

Mengden test-data som genereres angis av brukeren ved kjøring av
programmet ved å oppgi antallet datapunkter som argument. For eksempel
kan 1000 datapunkter genereres ved å kjøre \texttt{./aksjer\ 1000}, som
vil gi resultat som ligner følgende:

\begin{verbatim}
$ ./aksjer 1000
Optimal trade: Buy on day 689, then sell on day 803.
Gain: 158, time spent [µs]: 268
\end{verbatim}

Om det er ønskelig å se test-dataen som er generert for å manuelt
verifisere at algoritmen har ført til ønsket resultat kan man tilføye
argumentet \texttt{-s}. Om man for eksempel vil ha et lite datasett på
10 dager for å verifisere algoritmen kan man bruke
\texttt{./aksjer\ 10\ -s}. Dette vil gi utskrift som ligner følgende:

\begin{verbatim}
$ ./aksjer 10 -s
[3, -9, -6, 8, 4, -5, 1, -6, 4, -7]
Optimal trade: Buy on day 3, then sell on day 5.
Gain: 12, time spent [µs]: 0
\end{verbatim}

\hypertarget{testing}{%
\section{Testing}\label{testing}}

For tidtaking av programmet er Rusts innebygde tid-modul brukt.
Tidtaking startes etter at alle nødvendige variable er deklarert men før
selve algoritmen startes, og avsluttes etter at algoritmen er ferdig men
før resultatet presenteres. På denne måten sikres det at programmet ikke
tar allokering av minne eller skriving til stdout med i tidtakingen.

Testing har vist at mikrosekunder {[}µs{]} er en passende tidsenhet for
måling av kjøretid på egen maskin. For enkelhets skyld medfølger et lite
bash-skript, ``benchmark'', som kjører programmet 100 ganger med n =
1000 og deretter 100 ganger med n = 10000 og beregner gjennomsnittlig
kjøretid for testene. Et eksempel på utskrift av dette:

\begin{verbatim}
$ ./benchmark
No compiled binary found, compiling...
Compilation succeeded!

Running 100 tests with 1000 data points
Average time: 131 µs

Running 100 tests with 10000 data points
Average time: 11551 µs
\end{verbatim}

Dette underbygger tidligere analyse som tilsa at algoritmen har en
kompleksitet på O(n\^{}2), da en ti-dobling av datamengde fører til
omtrent en hundre-dobling av kjøretid.
